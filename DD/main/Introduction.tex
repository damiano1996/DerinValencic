\chapter{Introduction}


\section{Purpose}
The purpose of this document is describing the \gls{clup} system in a more specific way than the \gls{rasd}.
Therefore, it is suggested to read ahead the \gls{rasd}. This document is aimed both at stakeholders to view a preview of the final product, and at project managers as a guideline to follow.
It describes the system by both showing high-level and in-depth analysis of it, and how these analyses affected the architectural design choices made. It also shows the division of the system in components describing how they work and interact, how they are necessary for the fulfillment of the requirements and how they will be implemented and in which order.


\section{Scope}
Our application helps stores to prevent the diffusion of a virus, in a global pandemic situation, keeping customers in safety conditions.
One of means to contribute to the success of this goal is to guarantee the absence of crowds.
Considered one store, and established the maximum number of people that are allowed to be inside, our application is an instrument to keep the influx of people below that threshold.
It works managing automatically the influx of people inside the store, staggering the entrances in the store by using some parameters to keep them safe.
This is done by offering to customers a service that consist in a digital lining up system. It is designed to work remotely, but the possibility to line up physically is offered too. In this way, it should help to avoid crowds outside the store.
Moreover, to reach the largest number of people, this application should be easy to use and its functions should work consistently in time.
Instead, to the manager of the store it is given a dashboard where are shown significant data like the number of people currently in the store, the status of the queue, the influx of people during different intervals of time, etc. It also allows, when it is necessary, the store manager to make decisions that influences the flux of people (like for example, blocking the entrances for a period or modifying some parameters which the algorithm uses to manage the flux).
Furthermore, it is given to the customers a function that allow them to book a visit to the store specifying a time slot. This is thought to help people with limited availability during the day.
This option requires optionally at people to point out which product categories they are going to purchase or the departments they are going to visit. This is used to optimize the algorithm of the system to maximize the number of people inside the store during a certain time, by knowing their distribution. This can also be made automatically by using their data for long-time customers.


\section{Definitions, Acronyms, Abbreviations}

\subsection{Definitions}

\begin{tabularx}{\textwidth}{ >{\hsize=0.2\textwidth}X >{\hsize=0.8\textwidth}X}
    Customer      & a person who buys goods from the stores. We will use the term \textit{customers} to refer to natural persons, instead the term \textit{users} will be used to specify the virtual entity served by the application.                                               \\ \\
    Store Manager & a person who is in charge of the store. In our context, we assume that the \textit{store manager} controls the entrances to the store with the help of \gls{clup} service. In the real world scenario this activity can be delegated, without loss of generality.\\ \\
    Physical Spot & a digital device positioned outside the store that allows customers to obtain tickets to line up.\\ \\
    User & a virtual entity that interacts with the virtual service offered by \gls{clup}. The user can be a customers, a store manager and a physical spot (when it is acting as proxy). In case of ambiguous interpretations, we will specify the real entity name.\\ \\
\end{tabularx}

\begin{tabularx}{\textwidth}{ >{\hsize=0.2\textwidth}X >{\hsize=0.8\textwidth}X}
    Proxy          & an intermediary entity that exchanges information between two other entities. In our system, the physical spot can be seen as a proxy, since it allows customers to line up without the necessity to create an user account. From the point of view of the server, the physical spot is seen as an user. \\ \\
    Virtual Queue & a queue of users allocated in the memory of the server. When a user asks for a lining up operation, or a booking a visit operation, it is allocated in this queue.\\ \\
    Physical Queue & a queue of customers outside the store.\\ \\
    Ticket & a piece of paper or a virtual card given to customers to show that they have performed a lining up or a booking a visit operation.\\ \\
    QR code & a matrix composed by white and black squares encoding a string. It is reported on the ticket.         \\ \\
    System & we use this term to represent the entire service, composed by smartphone application and servers.\\ \\
    Application & program executable on smartphone.
\end{tabularx}

\subsection{Acronyms and Abbreviations}
\printglossary


\section{Revision History}

\begin{table}[H]
    \centering
    \begin{tabular}{ m{0.25\textwidth} | m{0.25\textwidth} }
        \textbf{Date} & \textbf{Modifications} \\
        \hline
        \today        & First version          \\
    \end{tabular}
    \caption{Revision history table.}
    \label{table:revisionHistory}
\end{table}


\section{Reference Documents}

\begin{itemize}
    \item Specification document: "R \& DD Assignment AY2020-2021.pdf".
    \item Slides of the lectures.
    \item \gls{rasd}: "CLup - Customers Line-up Requirements Analysis and Specification Document.pdf"
\end{itemize}


\section{Documents Structure}