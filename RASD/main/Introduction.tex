\chapter{Introduction}

\section{Purpose}

\subsection{Description of the Given Problem}

\subsection{Goals}

Bla bla bla...

Goals:
\begin{itemize}

	\item {\textbf{[G1]}}: Keep customers in safe condition w.r.t the \gls{dpcm} in force inside the store.
	
	% \item {\textbf{[G1]}}: Keep customers in safe condition w.r.t the \gls{dpcm} in force in the square outside the store.
		
	\item {\textbf{[G2]}}: Allow customers to line up from a remote device.
		
	\item {\textbf{[G3]}}: Allow store manager to monitor entrances.
	
	\item {\textbf{[G4]}}: Provide estimation of the waiting time.
	
	\item {\textbf{[G5]}}: Notify customers that their turn is coming.
	
	\item {\textbf{[G6]}}: Allow customers to line up from a physical spot.

	\item {\textbf{[G7]}}: Allow customers to book a visit from a remote device.

	% \item {\textbf{[G8]}}: Infer customers visits duration.
	
\end{itemize}

\section{Scope}

\section{Definitions, Acronyms, Abbreviations}

\subsection{Definitions}

\begin{tabularx}{\textwidth}{ >{\hsize=0.2\textwidth}X >{\hsize=0.8\textwidth}X}
  Customer & a person who buys goods from the stores. We will use the term \textit{customers} to refer to natural persons, instead the term \textit{users} will be used to specify the virtual entity served by the application.\\ \\
  Store Manager & a person who is in charge of the store. In our context, we assume that the \textit{store manager} controls the entrances to the store with the help of \gls{clup} service. In the real world scenario this activity can be delegated, without loss of generality.\\ \\
  Physical Spot & a digital device positioned outside the store that allows customers to obtain tickets to line up.\\ \\
  User & a virtual entity that interacts with the virtual service offered by \gls{clup}. The user can be a customers, a store manager and a physical spot (when it is acting as proxy). In case of ambiguous interpretations, we will specify the real entity name.\\ \\
  Proxy & an intermediary entity that exchanges information between two other entities. In our system, the physical spot can be seen as a proxy, since it allows customers to line up without the necessity to create an user account. From the point of view of the server, the physical spot is seen as an user.\\ \\
  Virtual Queue & a queue of users allocated in the memory of the server. When a user asks for a lining up operation, or a booking a visit operation, it is allocated in this queue.\\ \\
  Physical Queue & a queue of customers outside the store.\\ \\
  Ticket & a piece of paper or a virtual card given to customers to show that they have performed a lining up or a booking a visit operation.\\ \\
  QR code & a matrix composed by white and black squares encoding a string. It is reported on the ticket.\\ \\
  System & we use this term to represent the entire service, composed by smartphone application and servers.\\ \\
  Application & program executable on smartphones.
\end{tabularx}

\subsection{Acronyms and Abbreviations}
\printglossary

\section{Revision History}

\section{Reference Documents}

\section{Documents Structure}