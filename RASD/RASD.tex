\documentclass[a4paper,10pt,oneside,openany]{book}

\usepackage[english]{babel}

\usepackage[utf8]{inputenc} % Consente l'uso dei caratteri accentati italiani
\usepackage[T1]{fontenc} % Consente l'uso di caratteri speciali appartenenti in genere ad alfabeti non inglesi

\usepackage{fancyhdr} % Consente il rendering della copertina

\usepackage{graphicx} % Consente l'inserimento delle figure nel documento

\usepackage{subfigure} % Consente l'inserimento di sottofigure

\usepackage{tabularx} % Consente l'uso di funzionalità aggiuntive alle tabelle
\usepackage{longtable} % Consente la scrittura delle tabelle su più pagine

\usepackage[titletoc]{appendix} % Consente di porre le appendici in indice

\usepackage{float} % Necessario per inserire immagini flottanti

\usepackage{fancyvrb} % Consente l'utilizzo di finestre Verbatim (testo cche si vuole porre così com'è, senza formattazione alcuna) all'interno di una cornice

\usepackage{amsmath} % Consente l'uso di funzioni aggiuntive per le formule
\numberwithin{equation}{section}

\usepackage{hyperref} % Indice degli argomenti e rimandi cliccabili, per una navigazione più comoda all'interno del file

\usepackage{lipsum} % Consente di inserire testo fittizio all'interno della tesi. Inutile in fase di scrittura della tesi. Eliminare questa dicitura e tutte le diciture \lorem all'interno del testo.

\usepackage{listings} %Per il codice alloy

\usepackage[dvipsnames] {xcolor}%Per il codice alloy

\usepackage{alloy/alloy-style} %Per il codice alloy
\usepackage{rotating} % to rotate images

\usepackage{lineno}

\modulolinenumbers[1]

\usepackage[nopostdot,style=super,nonumberlist,toc, automake]{glossaries}
\makeglossaries
\renewcommand{\glossarysection}[2][]{}


\setacronymstyle{long-short}

\newacronym{clup}{CLup}{Customers Line-up}
\newacronym{dpcm}{d.P.C.m}{\textit{"decreto del Presidente del Consiglio dei ministri"}}
\newacronym{gps}{GPS}{Global Positioning System}
\newacronym{fifo}{FIFO}{First In First Out}
\newacronym{api}{API}{Application Programming Interface}
\newacronym{uml}{UML}{Unified Modeling Language}
\newacronym{qos}{QoS}{Quality of Service}
\newacronym{si}{SI}{International System of Units}
\newacronym{mvc}{MVC}{Model-View-Controller}
\newacronym{json}{JSON}{JavaScript Object Notation}
\newacronym{dbms}{DBMS}{Database Management System}
\newacronym{rasd}{RASD}{Requirements Analysis and Specification Document}
\newacronym{ux}{UX}{User Experience}
\newacronym{rest}{REST}{Representational State Transfer}
\newacronym{gui}{GUI}{Graphical User Interface}
\newacronym{dd}{DD}{Design Document} % input definitions


% Per inserire i contenuti, riferirsi ai file relativi e segnati tra parentesi graffe nei parametri \input

%*******************************************************
% Impostazioni della singola pagina e della copertina
%*******************************************************
\hypersetup{ % Visualizzazione dei collegamenti cliccabili
colorlinks,
citecolor=black,
filecolor=black,
linkcolor=black,
urlcolor=black
}

% Imposta l'intestazione delle pagine
\fancyhf{}
\pagestyle{fancy}
\lhead{}
\chead{\leftmark}
\rhead{}
\lfoot{}
\cfoot{}
\rfoot{\thepage}

%*******************************************************
% Copertina
%*******************************************************
% Imposta la copertina
% Non modificare questa parte
\usepackage[some]{background}
\SetBgScale{1}
\SetBgColor{gray}
\SetBgAngle{0}
\SetBgOpacity{0.07}
% Non modificare questa parte

% Titolo della tesi e autore
\title{CLup – Customers Line-up \\ Requirements Analysis and Specification Document}
\author{Damiano Derin}
\begin{document}
\pagenumbering{roman}
\begin{titlepage}
    \begin{center}
    % Insert the background here (front page)
    \BgThispage
    \includegraphics[scale=0.3]{images/polimi_logo.jpg}\\
    {\LARGE {\bfseries Politecnico di Milano \\}}
    \vspace{.5cm}
    {\Large {\bfseries Department of Computer Science and Engineering \\}}
    \vspace{1.0cm}
    
    {\Large {\bfseries Software Engineering 2 \\}}
    \vspace{2.0cm}
    
    
    {\LARGE {\bfseries CLup – Customers Line-up \\
    	Requirements Analysis \\ and \\ Specification Document\\}}
    \vspace{1cm}

    {\large \today \\
    }
    % Anno Accademico 2020/2021


    % Tabella per inserire i nomi del laureando, del relatore e dell'eventuale correlatore nel modo migliore all'interno della pagina
    \vfill
    \begin{table}[h]
        {\large
            \begin{tabular}{c c c c r c c | c c l}
                & & & & & & & & & Author \\
                & & & & & & & & & \bfseries Damiano Derin \\
                & & & & & & & & & \\
                & & & & & & & & & Author \\
                & & & & & & & & & \bfseries Jas Valencic \\
            \end{tabular}
        }
    \end{table}
    \vspace{1cm}
    \end{center}
\end{titlepage}


\frontmatter


%*******************************************************
% Indice
%*******************************************************
\tableofcontents


%*******************************************************
% Capitoli
%*******************************************************
\mainmatter

% Non modificare questa parte
\renewcommand{\chaptermark}[1]{\markboth{\MakeUppercase{\chaptername\ \thechapter.\ #1}}{}}
% Non modificare questa parte

% Inserire tutti i capitoli necessari. Si possono inserire tutti i capitoli
% all'interno di un singolo file .tex oppure un file .tex per ogni capitolo.

%\begin{linenumbers}
\chapter{Introduction}

\section{Purpose}

\subsection{Description of the Given Problem}

\subsection{Goals}

We identified the following goals:

\begin{itemize}

	\item {\textbf{[G1]}}: Keep customers in safe condition w.r.t the \gls{dpcm} in force inside the store.
		
	\item {\textbf{[G2]}}: Limit the physical line situation in the proximity of the store.
		
	\item {\textbf{[G3]}}: Allow customers to line up from a remote device.
	
	\item {\textbf{[G4]}}: Allow store manager to monitor entrances.
	
	\item {\textbf{[G5]}}: Allow customers to line up from a physical spot.
	
	\item {\textbf{[G6]}}: Allow customers to book a visit from a remote
device.

	
\end{itemize}

\section{Scope}

\section{Definitions, Acronyms, Abbreviations}

\subsection{Definitions}

\begin{tabularx}{\textwidth}{ >{\hsize=0.2\textwidth}X >{\hsize=0.8\textwidth}X}
  Customer & a person who buys goods from the stores. We will use the term \textit{customers} to refer to natural persons, instead the term \textit{users} will be used to specify the virtual entity served by the application.\\ \\
  Store Manager & a person who is in charge of the store. In our context, we assume that the \textit{store manager} controls the entrances to the store with the help of \gls{clup} service. In the real world scenario this activity can be delegated, without loss of generality.\\ \\
  Physical Spot & a digital device positioned outside the store that allows customers to obtain tickets to line up.\\ \\
  User & a virtual entity that interacts with the virtual service offered by \gls{clup}. The user can be a customers, a store manager and a physical spot (when it is acting as proxy). In case of ambiguous interpretations, we will specify the real entity name.\\ \\
  Proxy & an intermediary entity that exchanges information between two other entities. In our system, the physical spot can be seen as a proxy, since it allows customers to line up without the necessity to create an user account. From the point of view of the server, the physical spot is seen as an user.\\ \\
  Virtual Queue & a queue of users allocated in the memory of the server. When a user asks for a lining up operation, or a booking a visit operation, it is allocated in this queue.\\ \\
  Physical Queue & a queue of customers outside the store.\\ \\
  Ticket & a piece of paper or a virtual card given to customers to show that they have performed a lining up or a booking a visit operation.\\ \\
  QR code & a matrix composed by white and black squares encoding a string. It is reported on the ticket.\\ \\
  System & we use this term to represent the entire service, composed by smartphone application and servers.\\ \\
  Application & program executable on smartphone.
\end{tabularx}

\subsection{Acronyms and Abbreviations}
\printglossary

\section{Revision History}

\section{Reference Documents}

\section{Documents Structure}
\chapter{Overall Description}

\section{Product Perspective}

\section{Product Functions}

\section{User Characteristics}

\section{Assumptions, Dependencies and Constraints}
\chapter{Specific Requirements}

\section{External Interface Requirements}

\subsection{User Interfaces}

% screenshots here
\begin{figure}[h]
	\centering
	% \captionsetup{justification=centering,margin=1cm}

	\includegraphics[width=0.5\textwidth]{images/log_in.png}
	\caption{Log In page.}
	\label{customersUseCasesDiagram}
\end{figure}

\begin{figure}[h]
	\centering
	% \captionsetup{justification=centering,margin=1cm}

	\includegraphics[width=0.5\textwidth]{images/sign_up.png}
	\caption{Sign Up page.}
	\label{customersUseCasesDiagram}
\end{figure}


\subsection{Hardware Interfaces}
\subsection{Software interfaces}
\subsection{Communications Interfaces}

\section{Functional Requirements}

\subsection{Requirements}

\subsection{Definition of Use Case Diagrams}

Bla bla bla...

\begin{figure}[h]
	\centering
	% \captionsetup{justification=centering,margin=1cm}

	\includegraphics[width=1.0\textwidth]{images/customers_use_cases_diagram.pdf}
	\caption{Customers use cases diagram.}
	\label{customersUseCasesDiagram}
\end{figure}

\begin{table}[h!]
\centering
\begin{tabular}{| m{0.3\textwidth} | m{0.7\textwidth} |} 
	\hline
	\textbf{Name} & Sign Up \\ 
	\hline
	\textbf{Actor} & Customer \\ 
	\hline
	\textbf{Entry Conditions} & Customer is on the Sign Up page. \\ 
	\hline
	\textbf{Event Flows} &
	\begin{itemize}
		\item Customer inserts the requested information in the form.
		\item Customer clicks on the Sign Up button.
	\end{itemize} \\ 
	\hline
	\textbf{Exit Conditions} & Sign Up completed successfully and customer is logged in. \\ 
	\hline
	\textbf{Exceptions} &
	\begin{itemize}
		\item Customer's username already in use.
		\item Empty form field.
		\item Policy agreement rejected.
		\item Lost Internet connection.
	\end{itemize} \\ 
	\hline
\end{tabular}
\caption{Customer - use case: \textbf{Sign Up}.}
\label{tableSignUp}
\end{table}

\begin{table}[h!]
\centering
\begin{tabular}{| m{0.3\textwidth} | m{0.7\textwidth} |} 
	\hline
	\textbf{Name} & Log In \\ 
	\hline
	\textbf{Actor} & Customer \\ 
	\hline
	\textbf{Entry Conditions} & Customer is on the Log In page. \\ 
	\hline
	\textbf{Event Flows} &
	\begin{itemize}
	\item Customer inserts the requested information in the form.
	\item Customer clicks on the Log In button.
	\end{itemize} \\ 
	\hline
	\textbf{Exit Conditions} & Log In completed successfully and customer redirected to Home page. \\ 
	\hline
	\textbf{Exceptions} &
	\begin{itemize}
	\item Customer's username or password incorrect.
	\item Empty form field.
	\item Lost Internet connection.
	\end{itemize} \\ 
	\hline
\end{tabular}
\caption{Customer - use case: \textbf{Log In}.}
\label{tableLogIn}
\end{table}

\begin{table}[h!]
\centering
\begin{tabular}{| m{0.3\textwidth} | m{0.7\textwidth} |} 
	\hline
	\textbf{Name} & Log Out \\ 
	\hline
	\textbf{Actor} & Customer \\ 
	\hline
	\textbf{Entry Conditions} & Customer is on the Log Out page. \\ 
	\hline
	\textbf{Event Flows} &
	\begin{itemize}
	\item Customer clicks on the Log Out button.
	\end{itemize} \\ 
	\hline
	\textbf{Exit Conditions} & Log Out completed successfully and customer redirected to the Sign Up/Log In page. \\ 
	\hline
	\textbf{Exceptions} &
	\begin{itemize}
	\item Customer already logged out.
	\item Lost Internet connection.
	\end{itemize} \\ 
	\hline
\end{tabular}
\caption{Customer - use case: \textbf{Log Out}.}
\label{tableLogIn}
\end{table}

\begin{table}[h!]
\centering
\begin{tabular}{| m{0.3\textwidth} | m{0.7\textwidth} |} 
	\hline
	\textbf{Name} & Lining Up \\ 
	\hline
	\textbf{Actor} & Customer \\ 
	\hline
	\textbf{Entry Conditions} & Customer is on the Home page \\ 
	\hline
	\textbf{Event Flows} &
	\begin{itemize}
	\item Customer clicks on the Lining Up button.
	\end{itemize} \\ 
	\hline
	\textbf{Exit Conditions} & Lining Up completed successfully, the application returns, to the customer, the Status page and saves the QR code in the main memory. \\ 
	\hline
	\textbf{Exceptions} &
	\begin{itemize}
	\item Previous Lining Up action was not expired.
	\item Previous Booking Visit action was not expired.
	\item Customer wasn't logged.
	\item QR code cannot be saved correctly on the main memory.
	\item Lost Internet connection.
	\end{itemize} \\ 
	\hline
\end{tabular}
\caption{Customer - use case: \textbf{Lining Up}.}
\label{tableLogIn}
\end{table}

\begin{table}[h!]
\centering
\begin{tabular}{| m{0.3\textwidth} | m{0.7\textwidth} |} 
	\hline
	\textbf{Name} & Booking Visit \\ 
	\hline
	\textbf{Actor} & Customer \\ 
	\hline
	\textbf{Entry Conditions} & Customer is on the Home page \\ 
	\hline
	\textbf{Event Flows} &
	\begin{itemize}
	\item Customer clicks on the Booking Visit button.
	\item Customer fills the form with the requested data.
	\item Customer clicks on the Submit button.
	\end{itemize} \\ 
	\hline
	\textbf{Exit Conditions} & Booking Visit completed successfully and the application returns, to the customer, the Status page. \\ 
	\hline
	\textbf{Exceptions} &
	\begin{itemize}
	\item Previous Lining Up action was not expired.
	\item Previous Booking Visit action was not expired.
	\item Customer wasn't logged.
	\item QR code cannot be saved correctly on the main memory.
	\item Lost Internet connection.
	\end{itemize} \\ 
	\hline
\end{tabular}
\caption{Customer - use case: \textbf{Booking Visit}.}
\label{tableLogIn}
\end{table}

\begin{table}[h!]
\centering
\begin{tabular}{| m{0.3\textwidth} | m{0.7\textwidth} |} 
	\hline
	\textbf{Name} & Show QR code \\ 
	\hline
	\textbf{Actor} & Customer \\ 
	\hline
	\textbf{Entry Conditions} & Customer is on the Home page \\ 
	\hline
	\textbf{Event Flows} &
	\begin{itemize}
	\item Customer clicks on the Show QR code button.
	\end{itemize} \\ 
	\hline
	\textbf{Exit Conditions} & The application shows the QR code associated to the last Lining Up, or Booking Visit, operation. \\ 
	\hline
	\textbf{Exceptions} &
	\begin{itemize}
	\item QR code hasn't be saved on the application correctly.
	\item No Lining Up, or Booking Visit, action performed. % redirect to Lining Up page ot hide button if QR code doesn't exist.
	\item Customer wasn't logged.
	\end{itemize} \\ 
	\hline
\end{tabular}
\caption{Customer - use case: \textbf{Show QR code}.}
\label{tableLogIn}
\end{table}

\begin{table}[h!]
\centering
\begin{tabular}{| m{0.3\textwidth} | m{0.7\textwidth} |} 
	\hline
	\textbf{Name} & Get Status \\ 
	\hline
	\textbf{Actor} & Customer \\ 
	\hline
	\textbf{Entry Conditions} & Customer is on the Home page. \\ 
	\hline
	\textbf{Event Flows} &
	\begin{itemize}
	\item Customer clicks on the Get Status button.
	\end{itemize} \\ 
	\hline
	\textbf{Exit Conditions} & The application returns the Get Status page showing information about the last Lining Up, or Booking Visit, operation. \\ 
	\hline
	\textbf{Exceptions} &
	\begin{itemize}
	\item No operation previously performed, therefore there is no data to show.
	\item Customer wasn't logged.
	\item Lost Internet connection.
	\end{itemize} \\ 
	\hline
\end{tabular}
\caption{Customer - use case: \textbf{Get Status}.}
\label{tableLogIn}
\end{table}

\begin{table}[h!]
\centering
\begin{tabular}{| m{0.3\textwidth} | m{0.7\textwidth} |} 
	\hline
	\textbf{Name} & Print QR code \\ 
	\hline
	\textbf{Actor} & Customer \\ 
	\hline
	\textbf{Entry Conditions} & Customer is acting on the physical spot and he is on the Print QR code page. \\ 
	\hline
	\textbf{Event Flows} &
	\begin{itemize}
	\item Customer clicks on the Print QR code button.
	\end{itemize} \\ 
	\hline
	\textbf{Exit Conditions} & The spot prints the ticket with the QR code. \\ 
	\hline
	\textbf{Exceptions} &
	\begin{itemize}
	\item Spot finished the paper.
	\item Spot finished the ink.
	\item No more empty slots for the Lining Up in the current day.
	\end{itemize} \\ 
	\hline
\end{tabular}
\caption{Customer - use case: \textbf{Print QR code}.}
\label{tableLogIn}
\end{table}


\subsection{Use Cases and Sequence/Activity Diagrams}
\subsection{Mapping on Requirements}

\section{Performance Requirements}

\section{Design Constraints}

\subsection{Standard Compliance}
\subsection{Hardware limitations}
\subsection{Any Other Constraint}

\section{Software System Attributes}

\subsection{Reliability}
\subsection{Availability}
\subsection{Security}
\subsection{Maintainability}
\subsection{Portability}
\chapter{Formal Analysis Using Alloy}

\section{Alloy Code}

In this section is shown the alloy code that represent some parts of \gls{clup} model. The representation focus mainly on how the queue works with different kind of users, from the moment the line-up and so they appear inside this model to the moment they enter inside the store. Their place in the queue is represented by their ticket that encodes inside the QR code the information of lining-up and so of the user. That it is used to recognize the place that a person has in the queue and so to allow only the correct one to enter the store.
It is also shown by the Alloy’s graphic tool some worlds that evidence how the different components of the model and their properties work with each other in different context and some properties are checked by using the assertion feature. 


\lstinputlisting[language=Alloy]{alloy/Clup_final.als}

\begin{figure}[H]
	\centering
	\includegraphics[width=\textwidth]{images/Assertions.png}
	\caption{Assertions verification}
	\label{figure: Assertions verification}
\end{figure}

\begin{sidewaysfigure}
	\includegraphics[width=1.0\textwidth]{images/AlloyW1.png}
	\caption{World1.}
	\label{figure: World1}
\end{sidewaysfigure}

\begin{sidewaysfigure}
	\centering
	\includegraphics[width=1.0\textwidth]{images/world_3.png}
	\caption{World2.}
	\label{figure: World2}
\end{sidewaysfigure}

\begin{sidewaysfigure}
	\centering
	\includegraphics[width=1.0\textwidth]{images/EntraFunction}
	\caption{Entra function.}
	\label{figure: Entra function}
\end{sidewaysfigure}
\chapter{Effort Spent}

\begin{table}[H]
    \centering
    \begin{tabular}{| m{0.8\textwidth} | m{0.2\textwidth} |}
        \hline
        \textbf{Topic}                                                           & \textbf{Hours} \\
        \hline
        Preliminary Discussion                                                     & 6              \\
        \hline
        Introduction                                                               & 4              \\
        \hline
        Product Perspective                                                        & 6              \\
        \hline
        Product Functions                                                          & 1              \\
        \hline
        User Characteristics                                                       & 2              \\
        \hline
        Assumptions, Dependencies and Constraints                                  & 1              \\
        \hline
        External Interface Requirements                                            & 2              \\
        \hline
        Functional Requirements                                                    & 5              \\
        \hline
        Performance Requirements / Design Constraints / Software System Attributes & 5              \\
        \hline
        Alloy Code                                                                 & 12             \\
        \hline
        Revision                                                                   & 4              \\
        \hline
        \hline
        \textbf{Total:}                                                            & 48             \\
        \hline
    \end{tabular}
    \caption{Effort spent by Jas Valencic.}
\end{table}

\begin{table}[H]
    \centering
    \begin{tabular}{| m{0.8\textwidth} | m{0.2\textwidth} |}
        \hline
        \textbf{Topic}                                                           & \textbf{Hours} \\
        \hline
        Preliminary Discussion                                                     & 6              \\
        \hline
        Introduction                                                               & 3              \\
        \hline
        Product Perspective                                                        & 3              \\
        \hline
        Product Functions                                                          & 3              \\
        \hline
        User Characteristics                                                       & 1              \\
        \hline
        Assumptions, Dependencies and Constraints                                  & 2              \\
        \hline
        External Interface Requirements                                            & 8              \\
        \hline
        Functional Requirements                                                    & 10             \\
        \hline
        Performance Requirements / Design Constraints / Software System Attributes & 3              \\
        \hline
        Alloy Code                                                                 & 5              \\
        \hline
        Revision                                                                   & 4              \\
        \hline
        \hline
        \textbf{Total:}                                                            & 48             \\
        \hline
    \end{tabular}
    \caption{Effort spent by Damiano Derin.}
\end{table}
\chapter{References}

\begin{itemize}
    \item Specification document: "R \& DD Assignment AY2020-2021.pdf".
    \item Slides of the lectures.
    \item \glspl{rasd} of past students.
    \item Alloy documentation: "http://alloy.mit.edu/alloy/documentation.html"
    \item Google Maps services: "https://cloud.google.com/maps-platform"
    \item Google Firebase service: "https://firebase.google.com/"
\end{itemize}

%\end{linenumbers}


% bibliogarphy should be moved under References!
%*******************************************************
% Bibliografia
%*******************************************************
\nocite{*}
\bibliography{bib/bibliography}
\bibliographystyle{plain}


\end{document}
